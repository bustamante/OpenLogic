\documentclass[../../../include/open-logic-section]{subfiles}

\begin{document}

\olfileid{his}{set}{cantorplane}
\olsection{Cantor on the Line and the Plane}

Some of the circumstances surrounding the proof of
Schr\"oder-Bernstein tie in with the history we discussed in
\olref[his][set][pathology]{sec}. Recall that, in 1877,
Cantor proved that there are exactly as many points on a square as on
one of its sides. Here, we will present his (first attempted) proof.

Let $\unitline$ be the unit line, i.e., the set of points $[0,1]$. Let
$\unitsquare$ be the unit square, i.e., the set of points $\unitline
\times \unitline$. In these terms, Cantor proved that
$\cardeq{\unitline}{\unitsquare}$. He wrote a note to Dedekind,
essentially containing the following argument.

\begin{thm}\ollabel{thm:cantorplane}
$\cardeq{\unitline}{\unitsquare}$
\end{thm}

\begin{proof}[Proof: first part.] 
Fix $a, b \in \unitline$. Write them in binary notation, so that we
have infinite sequences of $0$s and $1$s, $a_1$, $a_2$, \dots, and
$b_1$, $b_2$, \dots, such that:
\begin{align*}
a &= 0.a_1a_2a_3a_4\dots\\
b &= 0.b_1b_2b_3b_4\dots
\intertext{Now consider the function $f \colon \unitsquare \to \unitline$ given by} 
f(a, b) & = 0.a_1b_1a_2b_2a_3b_3a_4b_4\dots
\end{align*}
Now $f$ is !!a{injection}, since if $f(a, b) = f(c,d)$, then  $a_n =
c_n$ and $b_n = d_n$ for all $n \in \Nat$, so that $a = c$ and $b =
d$.
\end{proof}

Unfortunately, as Dedekind pointed out to Cantor, this does not answer
the original question. Consider $0.\dot{1}\dot{0} =
0.1010101010\ldots$. We need that $f(a,b) = 0.\dot{1}\dot{0}$, where:
\begin{align*}
a&= 0.\dot{1}\dot{1} = 0.111111\ldots\\
b&= 0
\end{align*}
But $a = 0.\dot{1}\dot{1} = 1$. So, when we say ``write $a$ and $b$ in
binary notation'', we have to choose \emph{which} notation to use;
and, since $f$ is to be a \emph{function}, we can use only \emph{one}
of the two possible notations. But if, for example, we use the simple
notation, and write $a$ as ``$1.000\ldots$'', then we have no pair
$\tuple{a, b}$ such that $f(a, b) = 0.\dot{1}\dot{0}$. 

To summarise: Dedekind pointed out that, given the possibility of
certain recurring decimal expansions, Cantor's function $f$ is
!!a{injection} but \emph{not} !!a{surjection}. So Cantor has shown
only that $\cardle{\unitsquare}{\unitline}$ and \emph{not} that
$\cardeq{\unitsquare}{\unitline}$. 

Cantor wrote back to Dedekind almost immediately, essentially
suggesting that the proof could be completed as follows:

\begin{proof}[Proof: completed.] 
So, we have shown that $\cardle{\unitsquare}{\unitline}$. But there is
obviously !!a{injection} from $\unitline$ to $\unitsquare$: just lay
the line flat along one side of the square. So
$\cardle{\unitline}{\unitsquare}$ and
$\cardle{\unitsquare}{\unitline}$. By Schr\"{o}der--Bernstein
(\olref[sfr][siz][sb]{thm:schroder-bernstein}),
$\cardeq{\unitline}{\unitsquare}$.
\end{proof}

But of course, Cantor could not complete the last line in these terms,
for the Schr\"{o}der-Bernstein Theorem was not yet proved. Indeed,
although Cantor would subsequently formulate this as a general
conjecture, it was not satisfactorily proved until 1897. (And so,
later in 1877, Cantor offered a different proof of
\olref{thm:cantorplane}, which did not go via
Schr\"{o}der--Bernstein.)

\end{document}