% Part: Sets, Relations, Functions
% From projeto: THE OPEN LOGIC TEXT
% 01/06/2020 - LUIS HENRIQUE BUSTAMANTE

\documentclass[../../include/open-logic-part]{subfiles}

\begin{document}

\olpart{Conjuntos, Relações e Funções}
%\part{Sets, Relations, Functions}
%\olpart{sfr}{Sets, Relations, Functions}

%\begin{editorial}
  %The material in this part is a reasonably complete introduction to basic naive set theory. Unless students can be assumed to have this background, it's probably advisable to start a course with a review of this material, at least the part on sets, functions, and relations. This should ensure that all students have the basic facility with mathematical notation required for any of the other logical sections. NB: This part does not cover induction directly.
%%  O material nesta parte é uma introdução razoavelmente completa à teoria ingênua dos conjuntos. A menos que se suponha que o estudante tenha esse conhecimento prévio, é provadamente recomendado iniciar o curso com uma revisão desse material, pelo menos a parte sobre conjuntos, funções e relações. Isso deve garantir que todos os estudantes tenham o domínio básico com a notação matemática necessária para quaisquer outras seções de lógica. 
  %NB: Esta parte não cobre a indução diretamente
  
  %The presentation here would benefit from additional examples, especially, ``real life'' examples of relations of interest to the audience.
%%  A apresentação aqui será beneficiada de exemplos adicionais, especialmente, exemplos de relações ``da vida real'' de interesse do público.
  %It is planned to expand this part to cover naive set theory more extensively.
%%  Existe o plano de expandir essa seção para cobrir a teoria dos conjuntos mais extensivamente.
%\end{editorial}


\begin{editorial}
  This file includes the first four chapters of the part on Na\"ive
  Set Theory, in the original slowpaced version. The complete part is
  included by \verb|sets-functions-relations-complete.tex|. The
  material in this part is a reasonably complete introduction to basic
  naive set theory. Unless students can be assumed to have this
  background, it's probably advisable to start a course with a review
  of this material, at least the part on sets, functions, and
  relations. This should ensure that all students have the basic
  facility with mathematical notation required for any of the other
  logical sections.
\end{editorial}

\olimport[sets]{sets}

\olimport[relations]{relations}

\olimport[functions]{functions}

%\olimport[size-of-sets]{size-of-sets}

\OLEndPartHook
\end{document}
