% Part: sets-functions-relations
% Chapter: sets
% Section: proofs-about-sets

\documentclass[../../../include/open-logic-section]{subfiles}

\begin{document}

\olfileid{sfr}{set}{prf}
\olsection{Proofs about Sets}

\begin{editorial}
This section is superceded by \verb|content/method/proofs| and has
been removed from this chapter by default.
\end{editorial}

\begin{explain}
Sets and the notations we've introduced so far provide us with
convenient shorthands for specifying sets and expressing relationships
between them. Often it will also be necessary to prove claims about
such relationships. If you're not familiar with mathematical proofs,
this may be new to you. So we'll walk through a simple example. We'll
prove that for any sets $X$ and $Y$, it's always the case that $X \cap
(X \cup Y) = X$. How do you prove an identity between sets like this?
Recall that sets are determined solely by their !!{element}s, i.e.,
sets are identical iff they have the same !!{element}s. So in this
case we have to prove that (a) every !!{element} of $X \cap (X \cup
Y)$ is also an !!{element} of $X$ and, conversely, that (b) every
!!{element} of $X$ is also an !!{element} of $X \cap (X \cup Y)$. In
other words, we show that both (a) $X \cap (X \cup Y) \subseteq X$ and
(b) $X \subseteq X \cap (X \cup Y)$.

A proof of a general claim like ``every !!{element}~$z$ of $X \cap (X
\cup Y)$ is also an !!{element} of $X$'' is proved by first assuming
that an arbitrary $z \in X \cap (X \cup Y)$ is given, and proving from
this assumtion that $z \in X$. You may know this pattern as ``general
conditional proof.''  In this proof we'll also have to make use of the
definitions involved in the assumption and conclusion, e.g., in this
case of ``$\cap$'' and ``$\cup$.''  So case (a) would be argued as
follows:

\begin{quote}
(a) We first want to show that $X \cap (X \cup Y) \subseteq X$, i.e.,
by definition of $\subseteq$, that if $z \in X \cap (X \cup Y)$ then
$z \in X$, for any~$z$. So assume that $z \in X \cap (X \cup
Y)$. Since $z$ is an !!{element} of the intersection of two sets iff
it is an !!{element} of both sets, we can conclude that $z \in X$ and
also $z \in X \cup Y$. In particular, $z \in X$, which is what
we wanted to show.
\end{quote}

This completes the first half of the proof. Note that in the last
step we used the fact that if a conjunction ($z \in X$ and $z \in X
\cup Y$) follows from an assumption, each conjunct follows from that
same assumption. You may know this rule as ``conjunction
elimination,'' or $\land$Elim. Now let's prove (b):

\begin{quote}
(b) We now prove that $X \subseteq X \cap (X \cup Y)$, i.e., by
definition of $\subseteq$, that if $z \in X$ then also $z \in X \cap
(X \cup Y)$, for any~$z$. Assume $z \in X$. To show that $z \in X
\cap (X \cup Y)$, we have to show (by definition of ``$\cap$'') that
(i) $z \in X$ and also (ii) $z \in X \cup Y$. Here (i) is just our
assumption, so there is nothing further to prove. For (ii), recall that $z$
is an !!{element} of a union of sets iff it is an !!{element} of at least
one of those sets. Since $z \in X$, and $X \cup Y$ is the union of
$X$ and $Y$, this is the case here. So $z \in X \cup Y$. We've shown
both (i) $z \in X$ and (ii) $z \in X \cup Y$, hence, by definition of
``$\cap$,'' $z \in X \cap (X \cup Y)$.
\end{quote}

This was somewhat long-winded, but it illustrates how we reason about
sets and their relationships. We usually aren't this explicit; in
particular, we might not repeat all the definitions. A proof of our
result in a more advanced text would be much more compressed. It might
look something like this.
\end{explain}


\begin{prop}[Absorption]
For all sets $X$, $Y$,
\[
X \cap (X \cup Y) = X
\]
\end{prop}

\begin{proof}
(a) Suppose $z \in X \cap (X \cup Y)$. Then $z \in X$, so $X \cap (X
  \cup Y) \subseteq X$.

(b) Now suppose $z \in X$. Then also $z \in X \cup Y$, and therefore
  also $z \in X \cap (X \cup Y)$. Thus, $X \subseteq X \cap (X \cup
  Y)$.
\end{proof}

\begin{prob}
Prove in detail that $X \cup (X \cap Y) = X$. Then give a shortened,
compressed proof. (Hint: for the $X \cup (X \cap Y) \subseteq X$
direction you will need proof by cases, aka $\lor$Elim.)
\end{prob}

\end{document}
