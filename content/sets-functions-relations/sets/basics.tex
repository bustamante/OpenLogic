% Part: sets-functions-relations
% Chapter: sets
% Section: basics

\documentclass[../../../include/open-logic-section]{subfiles}


\begin{document}

\olfileid{sfr}{set}{bas}
%\olsection{Basics}

\olsection{Preliminares}

\begin{explain}
Conjuntos são as peças mais fundamentais entre os objetos matemáticos.
De fato, quase todo objeto matemático pode ser visto como um 
conjunto de algum tipo. Em lógica, como em outras partes da matemática, 
conjuntos e a fala teórica sobre conjuntos é ubíqua. Logo será importante
discutir o que são conjuntos, e introduzir as notações necessárias
para falar de conjuntos e operações sobre conjuntos de uma maneira padrão.
%Sets are the most fundamental building blocks of mathematical
%objects. In fact, almost every mathematical object can be seen as a
%set of some kind. In logic, as in other parts of mathematics, sets
%and set-theoretical talk is ubiquitous. So it will be important to
%discuss what sets are, and introduce the notations necessary to talk
%about sets and operations on sets in a standard way.
\end{explain}

\begin{defn}[Conjunto]
Um \emph{conjunto} é uma coleção de objetos, considerado independentemente
da maneira como é especificado, da ordem dos objetos no conjunto ou
da multiplicidade. Os objetos do conjunto são chamados de \emph{elementos}
ou \emph{membros} do conjunto. Se $a$ é um !!{elemento} de um conjunto~$X$,
nós escrevemos $a \in X$ (caso contrário, $a \notin X$). O conjunto
que não tem !!{elemento}s é chamado conjunto \emph{vazio} e denotado
pelo símbolo~$\emptyset$.
\end{defn}

%\begin{defn}[Set]
%A \emph{set} is a collection of objects, considered independently of
%the way it is specified, of the order of the objects in the set, or of
%their multiplicity. The objects making up the set are called
%\emph{elements} or \emph{members} of the set. If $a$ is !!a{element}
%of a set~$X$, we write $a \in X$ (otherwise, $a \notin X$). The set
%which has no !!{element}s is called the \emph{empty} set and denoted
%by the symbol~$\emptyset$.
%\end{defn}

\begin{ex}
Sempre que você tiver uma quantidade de objetos, você pode coletá-los juntos
em um conjunto. O conjunto do irmãos do Ricardo, por exemplo, é um conjunto
que contém uma pessoa, e nós podemos escrevê-lo como $S=\{\textrm{Ruth}\}$.
Em geral, quando nós temos alguns objetos $a_{1}$, \dots, $a_{n}$, então
o conjunto consistindo exatamente desses objetos é escrito $\{
a_{1}, \dots, a_{n}\}$. Frequentemente, nós iremos especificar um conjunto
por alguma propriedade que seus !!{elemento}s compartilham---como nós acabamos
de fazer, por exemplo, pela especificação $S$ como um conjunto de irmãos do
Ricardo. Nós iremos usar a sequinte notação simplificada para isso: 
$\Setabs{x}{\ldots x\ldots}$, onde o $\ldots x\ldots$ %stands
pela propriedade que $x$ tem que ter emm ordem de ser considerado como
!!{elemento}s do conjunto. No nossoexemplo, nós podemos especificar $S$
também como
\[
S = \Setabs{x}{x \text{ é o conjunto dos irmãos do Ricardo}}.
\]
\end{ex}

%\begin{ex}
%Whenever you have a bunch of objects, you can collect them together in
%a set. The set of Richard's siblings, for instance, is a set that
%contains one person, and we could write it as $S=\{\textrm{Ruth}\}$.
%In general, when we have some objects $a_{1}$, \dots, $a_{n}$, then
%the set consisting of exactly those objects is written $\{
%a_{1}, \dots, a_{n}\}$. Frequently we'll specify a set by some
%property that its !!{element}s share---as we just did, for instance, by
%specifying $S$ as the set of Richard's siblings. We'll use the
%following shorthand notation for that: $\Setabs{x}{\ldots x\ldots}$,
%where the $\ldots x\ldots$ stands for the property that $x$ has to
%have in order to be counted among the !!{element}s of the set. In our
%example, we could have specified $S$ also as
%\[
%S = \Setabs{x}{x \text{ is a sibling of Richard}}.
%\]
%\end{ex}

\begin{explain}
Quando nós dizemos que os conjuntos são independentes da maneira
que eles são especificados, nós queremos dizer que os !!{elemento}s de
um conjunto são tudo o que importam, Por exemplo, Acontece que
\begin{align*}
  & \{\text{Nicole}, \text{Jacob}\},\\
  & \Setabs{x}{\text{é uma sobrinha ou sobrinho do Ricardo}}, \text{ e}\\
  & \Setabs{x}{\text{é uma criança de Rute}}
\end{align*}
são três maneiras de especificar um e o mesmo conjunto.

%\begin{explain}
%When we say that sets are independent of the way they are specified,
%we mean that the !!{element}s of a set are all that matters. For instance,
%it so happens that
%\begin{align*}
%  & \{\text{Nicole}, \text{Jacob}\},\\
%  & \Setabs{x}{\text{is a niece or nephew of Richard}}, \text{ and}\\
%  & \Setabs{x}{\text{is a child of Ruth}}
%\end{align*}
%are three ways of specifying one and the same set.

Dizer que os conjuntos são considerados, independentemente da ordem dos seus
!!{elemento}s e suas multiplicidades é uma boa maneira de dizer que
\begin{align*}
  & \{\text{Nicole}, \text{Jacob}\} \text{ e}\\
  & \{\text{Jacob}, \text{Nicole}\}
\end{align*}
são duas maneiras de especificar o mesmo conjunto; e que
\begin{align*}
  & \{\text{Nicole}, \text{Jacob}\} \text{ e}\\
  & \{\text{Jacob}, \text{Nicole}, \text{Nicole}\}
\end{align*}
são também duas maneiras de especificar o mesmo conjunto. Em outras palavras, 
tudo o que importa são quais !!{elemento}s o conjunto tem. Os !!{elemento}s de um
conjunto não são ordenados e cada!!{elemento} ocorre somente uma vez. Quando nós
\emph{especificamos} ou \emph{escrevemos} um conjunto, !!{elemento}s podem ocorrer
multiplas vezes e em diferentes ordens, mas quaisquer descrições que somente
diferem na ordem dos !!{elemento}s ou em quantas vezes os !!{elemento}s são listados
descrevem o mesmo conjunto.
\end{explain}

%Saying that sets are considered independently of the order of their
%!!{element}s and their multiplicity is a fancy way of saying that
%\begin{align*}
%  & \{\text{Nicole}, \text{Jacob}\} \text{ and}\\
%  & \{\text{Jacob}, \text{Nicole}\}
%\end{align*}
%are two ways of specifying the same set; and that
%\begin{align*}
%  & \{\text{Nicole}, \text{Jacob}\} \text{ and}\\
%  & \{\text{Jacob}, \text{Nicole}, \text{Nicole}\}
%\end{align*}
%are also two ways of specifying the same set. In other words, all
%that matters is which !!{element}s a set has. The !!{element}s of a
%set are not ordered and each !!{element} occurs only once. When we
%\emph{specify} or \emph{describe} a set, !!{element}s may occur
%multiple times and in different orders, but any descriptions that only
%differ in the order of !!{element}s or in how many times !!{element}s
%are listed describes the same set.
%\end{explain}


\begin{defn}[Extensionalidade]
  Se $X$ e $Y$ são conjuntos, então $X$ e $Y$ são \emph{idênticos}, $X =
  Y$, sse todo !!{elemento} de~$X$ é também um !!{elemento} de~$Y$, e
  vice-versa. %!!um{elemento}
\end{defn}

%\begin{defn}[Extensionality]
%  If $X$ and $Y$ are sets, then $X$ and $Y$ are \emph{identical}, $X =
%  Y$, iff every !!{element} of~$X$ is also !!a{element} of~$Y$, and
%  vice versa.
%\end{defn}

\begin{explain}
Extenionalidade nos dá uma maneira de mostrar que conjuntos são idênticos:
para mostrar que $X = Y$, mostre que sempre quando $x \in X$ então $x \in Y$,
e sempre quando $y \in Y$ teremos $y \in X$.
\end{explain}

%\begin{explain}
%Extensionality gives us a way for showing that sets are identical: to
%show that $X = Y$, show that whenever $x \in X$ then also $x \in Y$,
%and whenever $y \in Y$ then also $y \in X$.
%\end{explain}

\begin{prob}
Mostre que existe somente um único conjunto vazio, i.e., mostre que se $X$
e $Y$ são conjuntos sem membros, então $X = Y$.
\end{prob}

%\begin{prob}
%Show that there is only one empty set, i.e., show that if $X$ and $Y$
%are sets without members, then $X = Y$.
%\end{prob}

\end{document}
