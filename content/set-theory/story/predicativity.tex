\documentclass[../../../include/open-logic-section]{subfiles}

\begin{document}

\olfileid{sth}{story}{predicative}	

\olsection{Predicative and Impredicative}

The Russell set, $R$, was defined by the formula $\Setabs{x}{x \notin
x}$. Spelled out more fully, $R$ would be the set which contains all
and only those sets which are not non-self-membered. So in defining
$R$, we quantify over the domain which would contain $R$ (if it
existed).

This is an \emph{impredicative} definition. More generally, we might
say that a definition is impredicative iff it quantifies over a domain
which contains the object that is being defined.  
	
In the wake of the paradoxes, Whitehead, Russell, Poincar\'{e} and
Weyl rejected such impredicative definitions as ``viciously
circular'':
\begin{quote}
	An analysis of the paradoxes to be avoided shows that they all
	result from a kind of vicious circle. The vicious circles in
	question arise from supposing that a collection of objects may
	contain members which can only be defined by means of the
	collection as a whole[\ldots. \textparagraph]

	The principle which enables us to avoid illegitimate totalities
	may be stated as follows: `Whatever involves \emph{all} of a
	collection must not be one of the collection'; or, conversely:
	`If, provided a certain collection had a total, it would have
	members only definable in terms of that total, then the said
	collection has no total.' We shall call this the `vicious-circle
	principle,' because it enables us to avoid the vicious circles
	involved in the assumption of illegitimate totalities.
	\citep[p.~37]{WhiteheadRussell1910}
\end{quote}
If we follow them in rejecting the \emph{vicious-circle principle},
then we might attempt to replace the disastrous Na\"{i}ve
Comprehension Scheme with something like the following:

	\
	\\\emph{Predicative Comprehension.} For every formula $\phi$ quantifying only over sets: the set$^\prime$ $\Setabs{x}{\phi(x)}$ exists.
	
\
\\So long as sets$^{\prime}$ are not sets, no contradiction will ensue.  

Unfortunately, Predicative Comprehension is not very
\emph{comprehensive}. After all, it introduces us to new entities,
sets$^\prime$. So we will have to consider formulas which quantify
over sets$^\prime$. If they always yield a set$^\prime$, then
Russell's paradox will arise again, just by considering the
set$^\prime$ of all non-self-membered sets$^\prime$. So, pursuing the
same thought, we must say that a formula quantifying over
sets$^\prime$ yields a corresponding set$^{\prime\prime}$. And then we
will need sets$^{\prime\prime\prime}$,
sets$^{\prime\prime\prime\prime}$, etc. To prevent a rash of primes,
it will be easier to think of these as sets$_0$, sets$_1$, sets$_2$,
sets$_3$, sets$_4$,\ldots. And this would give us a way into the
(simple) theory of types. 

There are a few obvious objections against such a theory (though it is
not obvious that they are \emph{overwhelming} objections). In brief:
the resulting theory is cumbersome to use; it is profligate in
postulating different kinds of objects; and it is not clear, in the
end, that impredicative definitions are even  \emph{all that bad}. 
	
To bring out the last point, consider this remark from
\citeauthor{Ramsey1925}:
\begin{quote}
	we may refer to a man as the tallest in a group, thus identifying
	him by means of a totality of which he is himself a member without
	there being any vicious circle. \citep{Ramsey1925}
\end{quote}
Ramsey's point is that ``the tallest man in the group'' \emph{is} an
impredicative definition; but it is obviously perfectly kosher. 

One might respond that, in this case, we could pick out the tallest
person by \emph{predicative} means. For example, maybe we could just
point at the man in question. The objection against impredicative
definitions, then, would clearly need to be limited to entities which
can \emph{only} be picked out impredicatively. But even then, we would
need to hear more, about why such ``essential impredicativity'' would
be so bad.\footnote{For more, see \citet{Linnebo2010}.}

Admittedly, impredicative definitions are extremely bad news, if we
want our definitions to provide us with something like a recipe for
\emph{creating} an object. For, given an impredicative definition, one
would genuinely be caught in a vicious circle: to create the
impredicatively specified object, one would \emph{first} need to
create all the objects (including the impredicatively specified
object), since the impredicatively specified object is specified in
terms of all the objects; so one would need to create the
impredicatively specified object before one had created it itself. But
again, this is only a serious objection against ``essentially
impredicatively'' specified sets, if we think of sets as things that
we \emph{create}. And we (probably) don't.

As such---for better or worse---the approach which became common does
not involve taking a hard line concerning (im)\-pre\-di\-ca\-tiv\-ity.
Rather, it involves what is now regarded as the cumulative-iterative
approach. In the end, this will allow us to stratify our sets into
``stages''---a \emph{bit} like the predicative approach stratifies
entities into sets$_0$, sets$_1$, sets$_2$, \ldots---but we will not
postulate any difference in kind between them. 

\end{document}