\documentclass[../../../include/open-logic-section]{subfiles}

\begin{document}

\olfileid{sth}{spine}{recursion}
\olsection{The Transfinite Recursion Theorem(s)}

The first thing we must do, though, is confirm that
\olref[valpha]{defValphas} is a successful definition. More generally,
we need to prove that any attempt to offer a transfinite by
(transfinite) recursion will succeed. That is the aim of this section.

\emph{Warning: this is very tricky material}. The overarching moral,
though, is quite simple: Transfinite Induction plus Replacement
guarantee the legitimacy of (several versions of) transfinite
recursion.

\begin{thm}[Bounded Recursion]\ollabel{transrecursionfun}
For any term $\tau(x)$ and any ordinal $\alpha$,\footnote{The term may
have parameters.} there is a unique function $f$ with domain $\alpha$
such that $(\forall \beta \in \alpha)f(\beta) =
\tau(\funrestrictionto{f}{\beta})$
\end{thm}

\begin{proof}
We will show that, for any $\delta \leq \alpha$, there is a unique
$g_\delta$ with domain $\delta$ such that $(\forall \beta \in
\delta)g(\beta) = \tau(\funrestrictionto{g}{\beta})$. 

We first establish uniqueness. Given $g_{\delta_1}$ and
$g_{\delta_2}$, a transfinite induction on their arguments shows that
$g(\beta) = h(\beta)$ for any $\beta \in \dom{g} \cap \dom{h} =
\delta_1 \cap \delta_2 = \min(\delta_1, \delta_2)$. So our functions
are unique (if they exist), and agree on all values.

To establish existence, we now use a simple transfinite induction
(\olref[ordinals][opps]{simpletransrecursion}) on ordinals
$\delta \leq \alpha$. 

Let $g_\emptyset = \emptyset$; this trivially behaves correctly.

Given $g_\delta$, let $g_{\ordsucc{\delta}} =  g_\delta \cup
\{\tuple{\delta, \tau(g_\delta)}\}$. This behaves correctly as
$\funrestrictionto{g_{\ordsucc{\delta}}}{\delta} = g_\delta$. 

Given $g_\gamma$ for all $\gamma \leq \delta$ with $\delta$ a limit
ordinal, let $g_\delta = \bigcup_{\gamma \in \delta} g_\gamma$. This
is a function, since our various $g_\beta$'s agree on all values. And
if $\beta \in \delta$ then $g_\delta(\beta) =
g_{\ordsucc{\beta}}(\beta) =
\tau(\funrestrictionto{g_{\ordsucc{\beta}}}{\beta}) =
\tau(\funrestrictionto{g_\delta}{\beta})$.

This completes the proof by transfinite induction. Now just let $f = g_\alpha$.
\end{proof}

If we allow ourselves to define a \emph{term} rather than a function,
then we can remove the bound $\alpha$ from the previous result. (In
the statement and proof of this result, when $\sigma$ is a term, we
let $\funrestrictionto{\sigma}{\alpha} = \Setabs{\tuple{\gamma,
\sigma(\gamma)}}{\gamma \in \alpha}$.)

\begin{thm}[General Recursion]\ollabel{transrecursionschema}
For any term $\tau(x)$ we can explicitly define a term
$\sigma(x)$,\footnote{Both terms may have parameters.} such that
$\sigma(\alpha) = \tau(\funrestrictionto{\sigma}{\alpha})$ for any
ordinal $\alpha$. 	
\end{thm}

\begin{proof}
For each $\alpha$, by \olref{transrecursionfun} are unique
$\ordsucc{\alpha}$-approximations, $f_{\ordsucc{\alpha}}$, and:
\[
	f_{\ordsucc{\alpha}}(\alpha) = 
	\tau(\funrestrictionto{f_{\ordsucc{\alpha}}}{\alpha}) = 
	\tau(\Setabs{\tuple{\gamma, f_{\ordsucc{\alpha}}(\gamma)}}{\gamma \in \alpha}).
\]
So define $\sigma(\alpha)$ as $f_{\ordsucc{\alpha}}(\alpha)$.
Repeating the induction of \olref{transrecursionfun}, but without the
upper bound, this is well-defined.
\end{proof}

Note that these results are \emph{schemas}. Crucially, we cannot
expect $\sigma$ to define a function, i.e., a certain kind of
\emph{set}, since then $\dom{\sigma}$ would be the set of all
ordinals, contradicting the Burali-Forti Paradox
(\olref[ordinals][basic]{buraliforti}).

It still remains to show, though, that \olref{transrecursionschema}
vindicates our definition of the $V_\alpha$s. This may not be
immediately obvious; but it will become apparent with a last, simple,
version of transfinite recursion.

\begin{thm}[Simple Recursion]\ollabel{simplerecursionschema} 
For any terms $\tau_1(x)$ and $\tau_2(x)$ and any set $A$, we can
explicitly define a term $\sigma(x)$ such that:\footnote{The terms may
have parameters.}
\begin{align*}
	\sigma(\emptyset) &= A\\
	\sigma(\ordsucc{\alpha}) &= \tau_1(\sigma(\alpha)) &&
		\text{for any ordinal }\alpha\\
	\sigma(\alpha) &= \tau_2(\ran{\funrestrictionto{\sigma}{\alpha}})&&
	\text{when }\alpha\text{ is a limit ordinal}
\end{align*}
\end{thm}

\begin{proof}
We start by defining a term, $\xi(x)$, as follows: 
%t $\xi(x) = A$ if $x$ is not a function whose domain is an ordinal;
%otherwise let:
\[
	\xi(x) = 
	\begin{cases}
			A &\text{if $x$ is not a function whose domain is an ordinal;}\\
			&\hspace{2em}\text{otherwise:}\\
			\tau_1(x(\alpha)) & \text{if $\dom{x} = \ordsucc{\alpha}$}\\
			\tau_2(\ran{x}) & \text{if $\dom{x}$ is a limit ordinal}
	\end{cases}
\]
By \olref{transrecursionschema}, there is a term $\sigma(x)$ such that
$\sigma(\alpha) = \xi(\funrestrictionto{\sigma}{\alpha})$ for every
ordinal $\alpha$; moreover, $\funrestrictionto{\sigma}{\alpha}$ is a
function with domain $\alpha$. We show that $\sigma$ has the required
properties, by simple transfinite induction
(\olref[ordinals][opps]{simpletransrecursion}). 

First, $\sigma(\emptyset) = \xi(\emptyset) = A$. 

Next, $\sigma(\ordsucc{\alpha}) = \xi(\funrestrictionto{\sigma}{\ordsucc{\alpha}}) = \tau_1(\funrestrictionto{\sigma}{\ordsucc{\alpha}}(\alpha)) = \tau_1(\sigma(\alpha))$.

Finally, if $\alpha$ is a limit ordinal, $\sigma(\alpha) = \xi(\funrestrictionto{\sigma}{\alpha}) = \tau_2(\ran{\funrestrictionto{\sigma}{\alpha}})$. % = \tau_2(\Setabs{\sigma(\gamma)}{\gamma\in \alpha})$.
\end{proof}

Now, to vindicate \olref[valpha]{defValphas}, just take $A
= \emptyset$ and $\tau_1(x) = \Pow{x}$ and $\tau_2(x) = \bigcup x$. So
we have vindicated the definition of the $V_\alpha$s!{}

\end{document}