\documentclass[../../../include/open-logic-section]{subfiles}

\begin{document}

\olfileid{sth}{spine}{rank}
\olsection{Rank}

Now that we have defined the stages as the $V_\alpha$'s, and we know
that every set is a subset of some stage, we can define the
\emph{rank} of a set. Intuitively, the rank of $A$ is the first moment
at which $A$ is formed. More precisely:

\begin{defn}\ollabel{defnsetrank}
For each set $A$, $\setrank{A}$ is the least $\alpha$ such that $A
\subseteq V_\alpha$.\footnote{Some books define $\setrank{A}$ as the
least $\alpha$ such that $A \in V_\alpha$. Since $A \subseteq V_\alpha
\liff A \in V_{\alpha+1}$, this is essentially just a notational
difference.}
\end{defn}

The well-ordering of ranks allows us to prove some important results:
\begin{prop}\ollabel{valphalowerrank}
For any ordinal $\alpha$, $V_\alpha = \Setabs{x}{\setrank{x} \in \alpha}$.
\end{prop}

\begin{proof}
If $\setrank{x} \in \alpha$ then $x \subseteq V_{\setrank{x}} \in
V_\alpha$, so $x \in V_\alpha$ as $V_\alpha$ is sublative (invoking
\olref[Valphabasic]{Valphabasicprops} multiple times). Conversely, by
definition of ``rank'' and Trichotomy on ordinals, if $\setrank{x}
\notin  \alpha$, then $x \nsubseteq V_\beta$ for any $\beta \in
\alpha$; and a simple transfinite induction on ordinals up to $\alpha$
shows that $x \notin V_\alpha$. 
\end{proof}

\begin{prop}\ollabel{rankmemberslower}
If $B \in A$, then $\setrank{B} \in \setrank{A}$.
\end{prop}

\begin{proof}
$A \subseteq V_{\setrank{A}} = \Setabs{x}{\setrank{x} \in
\setrank{A}}$ by \olref{valphalowerrank}.
\end{proof}

Using this fact, we can establish a result which allows us to prove
things about \emph{all sets} by a form of induction:

\begin{thm}[$\in$-Induction Scheme] 
For any formula $\phi$:\footnote{Which may have parameters}
\[
	\forall A((\forall x \in A)\phi(x) \lif \phi(A)) \lif \forall A \phi(A).
\]
\end{thm}

\begin{proof}
We will prove the contrapositive. So, suppose $\lnot \forall A
\phi(A)$. Since every set has a rank, Transfinite Induction
(\olref[ordinals][basic]{ordinductionschema}) tells us that there
is a non-$\phi$ of least possible rank. That is: there is some $A$
such that $\lnot \phi(A)$ and $\forall x(\setrank{x} \in \setrank{A}
\lif \phi(x))$. Now if $x \in A$ then $\setrank{x} \in
\setrank{A}$, by \olref{rankmemberslower}. So $(\forall x \in
A)\phi(x) \land \lnot \phi(A)$, falsifying the antecedent.
\end{proof}\noindent Here is an informal way to gloss this powerful
result. Say that $\phi$ is \emph{hereditary} iff whenever every
!!{element}s of a set is $\phi$, the set itself is $\phi$. Then
$\in$-Induction tells you the following: if $\phi$ is hereditary,
every set is $\phi$.

To wrap up the discussion of ranks (for now), we'll prove a few
claims which we have foreshadowed a few times. 

\begin{prop}\ollabel{ranksupstrict}
$\setrank{A} = \supstrict_{x \in A}\setrank{x}$.
\end{prop}

\begin{proof}
Let $\alpha = \supstrict_{x \in A}\setrank{x}$. By
\olref{rankmemberslower}, $\alpha \leq \setrank{A}$. But if $x \in A$
then $\setrank{x} \in \alpha$, so that $x \in V_\alpha$, and hence $A
\subseteq V_\alpha$, i.e., $\setrank{A} \leq \alpha$. Hence
$\setrank{A} = \alpha$.
\end{proof}

\begin{cor}\ollabel{ordsetrankalpha}
For any ordinal $\alpha$, $\setrank{\alpha} = \alpha$.
\end{cor}

\begin{proof}
Suppose for transfinite induction that $\setrank{\beta} = \beta$ for
all $\beta \in \alpha$. Now $\setrank{\alpha} = \supstrict_{\beta \in
\alpha}\setrank{\beta} = \supstrict_{\beta \in \alpha}\beta = \alpha$
by \olref{ranksupstrict}.
%	First note that $\setrank{\alpha} \neq \beta$ for any $\beta \in
%	\alpha$. For otherwise we would have $\beta = \setrank{\beta} \in
%	\setrank{\alpha} = \beta$ by \olref{rankmemberslower}, i.e.,
%	$\beta \in \beta$, a contradiction. 
%
%	Now note that $\alpha \subseteq V_\alpha$. For $\alpha =
%	\Setabs{\beta}{\beta \in \alpha}$ by
%	\olref[sfr][ordinals][basic]{ordissetofsmallerord}. And if
%	$\beta \in \alpha$ then $\beta \subseteq V_{\setrank{\beta}} =
%	V_\beta \in V_\alpha$, hence $\beta \in V_\alpha$, by
%	\olref[sfr][spine][Valphabasic]{Valphabasicprops} twice. 
\end{proof}

Finally, here is a quick proof of the result promised at the end of
\olref[foundation]{sec}, that $\ZFminus$ proves the conditional
$\emph{Regularity} \Rightarrow \emph{Foundation}$. (Note that the
notion of ``rank'' and \olref{rankmemberslower} are available for use
in this proof since---as mentioned at the start of this section---they
can be presented using $\ZFminus + \text{Regularity}$.)

\begin{prop}[working in $\ZFminus + \text{Regularity}$]\ollabel{zfminusregularityfoundation} Foundation holds.
\end{prop}

\begin{proof}
Fix $A \neq \emptyset$, and some $B \in A$ of least possible rank. If
$c \in B$ then $\setrank{c} \in \setrank{B}$ by
\olref{rankmemberslower}, so that $c \notin A$ by choice of $B$.
\end{proof}

\end{document}