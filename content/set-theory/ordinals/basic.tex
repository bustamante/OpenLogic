\documentclass[../../../include/open-logic-section]{subfiles}

\begin{document}

\olfileid{sth}{ordinals}{basic}
\olsection{Basic Properties of the Ordinals}

We observed that the first few ordinals are the natural numbers. The
main reason for developing a theory of ordinals is to extend the
principle of induction which holds on the natural numbers. We will
build up to this via a sequence of elementary results.

\begin{lem}\ollabel{ordmemberord}
Every !!{element} of an ordinal is an ordinal.
\end{lem}

\begin{proof}
Let $\alpha$ be an ordinal with $b \in \alpha$. Since $\alpha$ is
transitive, $b \subseteq \alpha$. So $\in$ well-orders $b$ as $\in$
well-orders $\alpha$.

For transitivity, suppose $x \in c \in b$. So $c \in \alpha$ as $b
\subseteq \alpha$. Again, as $\alpha$ is transitive, $c \subseteq
\alpha$, so that $x \in \alpha$. So $x, c, b \in \alpha$. But $\in$
well-orders $\alpha$, so that $\in$ is a transitive relation on
$\alpha$ by \olref[sth][ordinals][wo]{wo:strictorder}. So since $x
\in c \in b$, we have $x \in b$. Generalising, $c \subseteq b$
\end{proof}

\begin{cor}\ollabel{ordissetofsmallerord}
$\alpha = \Setabs{\beta \in \alpha}{\beta \text{ is an ordinal}}$, for
any ordinal~$\alpha$
\end{cor}

\begin{proof}
Immediate from \olref{ordmemberord}.
\end{proof}

The rough gist of the next two main results,
\olref{ordinductionschema} and \olref{ordtrichotomy}, is that the
ordinals themselves are well-ordered by membership:

\begin{thm}[Transfinite Induction]\ollabel{ordinductionschema}
For any formula $\phi(x)$:\footnote{The formula may have parameters,
which need not be ordinals.} 	
\[
	\text{if }\exists \alpha \phi(\alpha)\text{, then }\exists \alpha(\phi(\alpha)
	\land  (\forall \beta \in \alpha) \lnot \phi(\beta))
\]
where the displayed quantifiers are implicitly restricted to ordinals.
\end{thm}

\begin{proof}
%\olref{ordinductionschema1}. 
Suppose $\phi(\alpha)$, for some ordinal $\alpha$. If $ (\forall \beta
\in \alpha) \lnot \phi(\beta)$, then we are done. Otherwise, as
$\alpha$ is an ordinal, it has some $\in$-least !!{element} which is
$\phi$, and this is an ordinal by \olref{ordmemberord}.
%	\olref{ordinductionschema2}. Suppose there is some ordinal
%	$\gamma$ such that $\lnot\phi(\gamma)$. Then by
%	\olref{ordinductionschema1}, there is an $\in$-minimal ordinal
%	$\alpha$ for which $\lnot\phi(\alpha)$. So $(\forall \beta <
%	\alpha) \phi(\beta)$, rendering the antecedent of the conditional
%	false.
\end{proof}

Note that we can equally express \olref{ordinductionschema} as the
scheme:
\[
\text{if }\forall \alpha((\forall \beta \in \alpha)\phi(\beta) \lif 
\phi(\alpha))\text{, then }\forall \alpha\phi(\alpha)
\]
just by taking $\lnot\phi(\alpha)$ in \olref{ordinductionschema} and reasoning as in \olref[ordinals][wo]{propwoinduction}.

\begin{thm}[Trichotomy]\ollabel{ordtrichotomy} 
$\alpha \in \beta \lor \alpha = \beta \lor \beta \in \alpha$, for any
ordinals $\alpha$ and $\beta$. 
\end{thm}

\begin{proof}
The proof is by double induction, i.e., using
\olref{ordinductionschema} twice. Say that $x$ is \emph{comparable}
with $y$ iff $x \in y \lor x = y \lor y \in x$. 

For induction, suppose that every ordinal in~$\alpha$ is comparable
with \emph{every} ordinal. For further induction, suppose that
$\alpha$ is comparable with every ordinal in~$\beta$. We will show that
$\alpha$ is comparable with~$\beta$. By induction on~$\beta$, it will
follow that $\alpha$ is comparable with every ordinal; and so by
induction on~$\alpha$, \emph{every} ordinal is comparable with
\emph{every} ordinal, as required. 

It suffices to assume that $\alpha \notin \beta$ and $\beta \notin
\alpha$, and show that $\alpha = \beta$. 

To show that $\alpha \subseteq \beta$, fix $\gamma \in \alpha$; this
is an ordinal by \olref{ordmemberord}. So by the first induction
hypothesis, $\gamma$ is comparable with $\beta$. But if either $\gamma
= \beta$ or $\beta \in \gamma$ then $\beta \in \alpha$ (invoking the
fact that $\alpha$ is transitive if necessary), contrary to our
assumption; so $\gamma \in \beta$. Generalising, $\alpha \subseteq
\beta$.

Exactly similar reasoning, using the second induction hypothesis,
shows that $\beta \subseteq \alpha$. So $\alpha = \beta$.
\end{proof}\noindent As such, we will sometimes write $\alpha <\beta$
rather than $\alpha \in \beta$, since $\in$ is behaving as an ordering
relation. There are no deep reasons for this, beyond familiarity, and
because it is easier to write $\alpha \leq \beta$ than $\alpha \in
\beta \lor \alpha = \beta$.\footnote{We could write $\alpha
\mathrel{\underline{\in}} \beta$; but that would be wholly
non-standard.}

Here are two quick consequences of our last results, the first of
which puts our new notation into action:

\begin{cor}\ollabel{ordordered}
If $\exists \alpha\phi(\alpha)$, then $\exists \alpha(\phi(\alpha)
\land \forall \beta(\phi(\beta) \lif \alpha \leq \beta))$.
Moreover, for any ordinals $\alpha, \beta, \gamma$, both $\alpha
\notin \alpha$ and $\alpha \in \beta \in \gamma \lif \alpha \in
\gamma$. 
\end{cor}

\begin{proof}
Just like \olref[wo]{wo:strictorder}.
\end{proof}

\begin{prob}
Complete the ``exactly similar reasoning'' in the proof of
\olref[sth][ordinals][basic]{ordtrichotomy}.
\end{prob}

\begin{cor}\ollabel{corordtransitiveord}
$A$ is an ordinal iff $A$ is a transitive set of ordinals.
\end{cor}

\begin{proof}
\emph{Left-to-right.} By \olref{ordmemberord}. \emph{Right-to-left.}
If $A$ is a transitive set of ordinals, then $\in$ well-orders $A$ by
\olref{ordinductionschema} and \olref{ordtrichotomy}.
\end{proof}

But, although we have said that $\in$ well-orders the ordinals, we
have to be \emph{very cautious} about all this, thanks to the
following:

\begin{thm}[Burali-Forti Paradox]\ollabel{buraliforti}
There is no set of all the ordinals
\end{thm}

\begin{proof}
For reductio, suppose $O$ is the set of all ordinals. If $\alpha \in
\beta \in O$, then $\alpha$ is an ordinal, by \olref{ordmemberord}, so
$\alpha \in O$. So $O$ is transitive, and hence $O$ is an ordinal by
\olref{corordtransitiveord}. Hence $O \in O$, contradicting
\olref{ordordered}. 
\end{proof}

This result is named after \citeauthor{Burali-Forti1897}. But, as van
Heijenoort explains:
\begin{quote}
  Burali-Forti himself considered the contradiction as establishing,
  by \emph{reductio ad absurdum}, the result that the natural ordering
  of the ordinals is just a partial ordering.
  \citep[p.~105]{Heijenoort1967}
\end{quote}
It was Cantor in 1899---in a letter to Dedekind---who first saw
clearly the \emph{contradiction} in supposing that there is a set of
all the ordinals. (For further historical discussion, see
\citealt[p.~105]{Heijenoort1967}.)

To summarise, ordinals are sets which are individually well-ordered by
membership, and collectively well-ordered by membership. 

Rounding this off, here are some more basic properties about the
ordinals which follow from \olref{ordinductionschema} and
\olref{ordtrichotomy}. 

\begin{prop}
Any strictly descending sequence of ordinals is finite.
\end{prop}

\begin{proof}
Any infinite strictly descending sequence of ordinals $\ldots \in
\alpha_3 \in \alpha_2 \in \alpha_1 \in \alpha_0$ has no $\in$-minimal
member, contradicting \olref{ordinductionschema}.
\end{proof}

\begin{prop}\ollabel{ordinalsaresubsets}
$\alpha \subseteq \beta \lor \beta \subseteq \alpha$, for any ordinals
$\alpha, \beta$.
\end{prop}

\begin{proof}
If $\alpha \in \beta$, then $\alpha \subseteq \beta$ as $\beta$ is
transitive. Similarly, if $\beta \in \alpha$, then $\beta \subseteq
\alpha$. And if $\alpha = \beta$, then $\alpha \subseteq \beta$ and
$\beta \subseteq \alpha$. So by   \olref{ordtrichotomy} we are done.
\end{proof}

\begin{prop}\ollabel{ordisoidentity}
$\alpha = \beta$ iff $\ordeq{\alpha}{\beta}$, for any ordinals
$\alpha, \beta$.
\end{prop}

\begin{proof}
The ordinals are well-orders; so this is immediate from Trichotomy
(\olref[basic]{ordtrichotomy}) and
\olref[iso]{wellordnotinitial}. 
\end{proof}

\begin{prob}\ollabel{probunionordinalsordinal}
Prove that, if every member of $X$ is an ordinal, then $\bigcup X$ is an ordinal.
\end{prob}

\end{document}