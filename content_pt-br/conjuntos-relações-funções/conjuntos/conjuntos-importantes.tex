% Part: sets-functions-relations
% Chapter: sets
% Section: important-sets

\documentclass[../../../include/open-logic-section]{subfiles}

\begin{document}

\olfileid{sfr}{set}{set}

\olsection{Alguns conjuntos importantes}
%\olsection{Some Important Sets}

\begin{ex}
Na maioria das vezes, nós estaremos lidando com os conjuntos de
objetos matemáticos como membros. Você irá se lembrar de vários
conjuntos de números: $\Nat$ é o conjunto dos números \emph{naturais}
$\{0, 1, \allowbreak 2, 3, \dots\}$; $\Int$ o conjunto dos \emph{inteiros},
\[
\{\dots, -3, -2,
-1, 0, 1, 2, 3, \dots\};
\]
$\Rat$ o conjunto dos números \emph{racionais} 
($\Rat = \Setabs{z/n}{z \in \Int, n \in \Nat, n \neq 0}$); e
$\Real$ o conjunto dos números \emph{reais}. Todos esses são conjuntos
\emph{infinitos}, isto é, cada um deles contém infinitos !!{elemento}s.
Como se tornou, $\Nat$, $\Int$, $\Rat$ têm o mesmo número de 
de !!{elemento}s, enquanto $\Real$ possui uma porção maior---$\Nat$,
$\Int$, $\Rat$ são ``!!{enumeráveis} e infinitos'' enquanto
$\Real$ é ``!!{não-enumerável}''.

Nós iremos algumas vezes também usar o conjunto dos inteiros positivos
$\Int^+ = \{1, 2, 3, \dots\}$ e o conjunto contendo exatamente os dois 
primeiros números naturais $\Bin = \{0, 1\}$.
\end{ex}

%\begin{ex}
%Mostly we'll be dealing with sets that have mathematical objects as
%members. You will remember the various sets of numbers: $\Nat$
%is the set of \emph{natural} numbers $\{0, 1, \allowbreak 2, 3, \dots\}$;
%$\Int$ the set of \emph{integers},
%\[
%\{\dots, -3, -2,
%-1, 0, 1, 2, 3, \dots\};
%\]
%$\Rat$ the set of
%\emph{rational} numbers ($\Rat = \Setabs{z/n}{z \in \Int, n \in \Nat, n \neq 0}$); and
%$\Real$ the set of \emph{real} numbers. These are all \emph{infinite}
%sets, that is, they each have infinitely many !!{element}s. As it turns
%out, $\Nat$, $\Int$, $\Rat$ have the same number
%of !!{element}s, while $\Real$ has a whole bunch more---$\Nat$,
%$\Int$, $\Rat$ are ``!!{enumerable} and infinite'' whereas
%$\Real$ is ``!!{nonenumerable}''.

%We'll sometimes also use the set of positive integers $\Int^+ = \{1,
%2, 3, \dots\}$ and the set containing just the first two natural
%numbers $\Bin = \{0, 1\}$.
%\end{ex}

\begin{ex}[Strings]
Outro exemplo interessante é o conjunto $A^{*}$ de \emph{strings finitas}
sob um alfabeto $A$: qualquer sequência finita de elementos de $A$
é uma string sob $A$. Nós icluímos a \emph{string vazia $\Lambda$}
entre as strings sob $A$, para todo alfabeto~$A$. Por exemplo,
\begin{multline*}
\Bin^*
=\{\Lambda,0,1,00,01,10,11,\\
000,001,010,011,100,101,110,111,0000,\ldots\}.
\end{multline*}
Se $x=x_{1}\ldots x_{n}\in A^{*}$ é uma string consistindo de $n$
``letras'' de $A$, então nós dizemos que o \emph{tamanho} da string é~$n$
e escrevemos $\len{x}=n$.
\end{ex}

%\begin{ex}[Strings]
%Another interesting example  is the set $A^{*}$ of
%\emph{finite strings} over an alphabet $A$: any finite sequence of elements of
%$A$ is a string over $A$. We include the \emph{empty string $\Lambda$}
%among the strings over $A$, for every alphabet~$A$. For instance,
%\begin{multline*}
%\Bin^*
%=\{\Lambda,0,1,00,01,10,11,\\
%000,001,010,011,100,101,110,111,0000,\ldots\}.
%\end{multline*}
%If $x=x_{1}\ldots x_{n}\in A^{*}$is a string consisting of $n$
%``letters'' from $A$, then we say \emph{length} of the string is~$n$
%and write $\len{x}=n$.
%\end{ex}

\begin{ex}[Sequências infinitas]
Para qualquer conjunto $A$, nós podemos também considerar o conjunto~$A^\omega$
de sequências inficnitas de !!{elemento}s de~$A$. Uma sequência infinita 
$a_1a_2a_3a_4\dots$ consiste de uma lista infinitas de objetos em uma direção
(\emph{one-way}), cada um do qual é um !!{elemento} de~$A$.
\end{ex}

%\begin{ex}[Infinite sequences]
%For any set $A$ we may also consider the set~$A^\omega$ of infinite
%sequences of !!{element}s of~$A$. An infinite sequence
%$a_1a_2a_3a_4\dots$ consists of a one-way infinite list of objects,
%each one of which is !!a{element} of~$A$.
%\end{ex}

\end{document}