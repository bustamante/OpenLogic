% Part: sets-functions-relations
% Chapter: functions
% Section: inverses

\documentclass[../../../include/open-logic-section]{subfiles}

\begin{document}

\olfileid{sfr}{fun}{inv}
\olsection{Inverses of Functions}

\begin{explain}
We think of functions as maps. An obvious question to ask about
functions, then, is whether the mapping can be ``reversed.'' For
instance, the successor function $f(x) = x + 1$ can be reversed, in
the sense that the function $g(y) = y - 1$ ``undoes'' what $f$ does. 

But we must be careful. Although the definition of~$g$ defines a
function $\Int \to \Int$, it does not define a \emph{function} $\Nat
\to \Nat$, since $g(0) \notin \Nat$.  So even in simple cases, it is
not quite obvious whether a function can be reversed; it may depend on
the domain and codomain.

This is made more precise by the notion of an inverse of a function.
\end{explain}

\begin{defn}
A function $g \colon B \to A$ is an \emph{inverse} of a function $f
\colon A \to B$ if $f(g(y)) = y$ and $g(f(x)) = x$ for all $x \in A$
and $y \in B$.
\end{defn}

If $f$ has an inverse~$g$, we often write $f^{-1}$ instead of~$g$.

\begin{explain}
Now we will determine when functions have inverses. A good candidate
for an inverse of $f\colon A \to B$ is $g\colon B \to A$ ``defined
by''
\[
g(y) = \text{``the'' $x$ such that $f(x) = y$.}
\]
But the scare quotes around ``defined by'' (and ``the'') suggest that
this is not a definition.  At least, it will not always work, with
complete generality. For, in order for this definition to specify a
function, there has to be one and only one~$x$ such that $f(x) =
y$---the output of~$g$ has to be uniquely specified. Moreover, it has
to be specified for every $y \in B$.  If there are $x_1$ and $x_2 \in
A$ with $x_1 \neq x_2$ but $f(x_1) = f(x_2)$, then $g(y)$ would not be
uniquely specified for $y = f(x_1) = f(x_2)$. And if there is no~$x$
at all such that $f(x) = y$, then $g(y)$ is not specified at all.  In
other words, for $g$ to be defined, $f$ must be both !!{injective} and
!!{surjective}.
\end{explain}

\begin{prop}\ollabel{prop:bijection-inverse}
Every !!{bijection} has a unique inverse.
\end{prop}

\begin{proof}
Exercise.
\end{proof}

\begin{prob}
Prove \olref[sfr][fun][inv]{prop:bijection-inverse}. That is, show that if
$f\colon A \to B$ is !!{bijective}, an inverse $g$ of $f$ exists. You
have to define such a $g$, show that it is a function, and show that
it is an inverse of~$f$, i.e., $f(g(y)) = y$ and $g(f(x)) = x$ for all
$x \in A$ and $y \in B$.
\end{prob}

\begin{explain}
However, there is a slightly more general way to extract inverses. We
saw in \olref[kin]{sec} that every function $f$ induces
!!a{surjection} $f' \colon A \to \ran{f}$ by letting $f'(x) = f(x)$
for all $x \in A$. Clearly, if $f$ is !!a{injection}, then $f'$ is
!!a{bijection}, so that it has a unique inverse by
\olref{prop:bijection-inverse}. By a very minor abuse of notation, we
sometimes call the inverse of $f'$ simply ``the inverse of $f$.''
\end{explain}

\begin{prob}
Show that if $f\colon A \to B$ has an inverse~$g$, then $f$ is
!!{bijective}.
\end{prob}

\begin{prop}\ollabel{prop:inverse-unique}
Every function~$f$ has at most one inverse.
\end{prop}

\begin{proof}
Exercise.
\end{proof}

\begin{prob}
Prove \olref[sfr][fun][inv]{prop:inverse-unique}. That is, show that
if $g\colon B \to A$ and $g'\colon B \to A$ are inverses of~$f\colon A
\to B$, then $g = g'$, i.e., for all $y \in B$, $g(y) = g'(y)$.
\end{prob}

\end{document}
