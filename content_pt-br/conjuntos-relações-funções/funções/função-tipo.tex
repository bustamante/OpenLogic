% Part: sets-functions-relations
% Chapter: functions
% Section: kinds

\documentclass[../../../include/open-logic-section]{subfiles}

\begin{document}

\olfileid{sfr}{fun}{kin}
\olsection{Kinds of Functions}

\begin{explain}
It will be useful to introduce a kind of taxonomy for some of the
kinds of functions which we encounter most frequently. 

To start, we might want to consider functions which have the property
that every member of the codomain is a value of the function. Such
functions are called !!{surjective}, and can be pictured as in
\olref{fig:surjective}.

\begin{figure}
  \olasset{assets/diagrams/surjective.tikz}
  \caption{!!^a{surjective} function has every !!{element} of the
    codomain as a value.}
  \ollabel{fig:surjective}
\end{figure}
\end{explain}

\begin{defn}[!!^{surjective} function]
A function $f \colon A \rightarrow B$ is \emph{!!{surjective}} iff $B$
is also the range of~$f$, i.e., for every $y \in B$ there is at least
one $x \in A$ such that~$f(x) = y$, or in symbols:
\[
  (\forall y \in B)(\exists x \in A)f(x) = y.
\]
We call such a function !!a{surjection} from $A$ to $B$.
\end{defn}

\begin{explain}
If you want to show that $f$ is !!a{surjection}, then you need to show
that every object in $f$'s codomain is the value of $f(x)$ for some
input $x$.

Note that any function \emph{induces} !!a{surjection}. After all,
given a function $f \colon A \to B$, let $f' \colon A \to \ran{f}$ be
defined by $f'(x) = f(x)$. Since $\ran{f}$ is \emph{defined} as
$\Setabs{f(x) \in B}{x \in A}$, this function $f'$ is guaranteed to be
!!a{surjection}
\end{explain}

\begin{explain}
Now, any function maps each possible input to a unique output. But
there are also functions which never map different inputs to the same
outputs. Such functions are called !!{injective}, and can be pictured
as in \olref{fig:injective}.
\begin{figure}
  \olasset{assets/diagrams/injective.tikz}
  \caption{!!^a{injective} function never maps two different
    arguments to the same value.}
  \ollabel{fig:injective}
\end{figure}
\end{explain}

\begin{defn}[!!^{injective} function] 
A function $f \colon A \rightarrow B$ is \emph{!!{injective}} iff for
each $y \in B$ there is at most one $x \in A$ such that~$f(x) = y$. We
call such a function !!a{injection} from $A$ to~$B$.
\end{defn}

\begin{explain}
If you want to show that $f$ is !!a{injection}, you need to show that
for any !!{element}s $x$ and $y$ of $f$'s domain, if $f(x)=f(y)$, then
$x=y$. 
\end{explain}

\begin{ex}
The constant function $f\colon \Nat \to \Nat$ given by $f(x) = 1$ is
neither !!{injective}, nor !!{surjective}.

The identity function $f\colon \Nat \to \Nat$ given by $f(x) = x$ is
both !!{injective} and !!{surjective}.

The successor function $f \colon \Nat \to \Nat$ given by $f(x) = x+1$
is !!{injective} but not !!{surjective}.
  
The function $f \colon \Nat \to \Nat$ defined by:
\[
  f(x) =
  \begin{cases}
    \frac{x}{2} & \text{if $x$ is even} \\
    \frac{x+1}{2} & \text{if $x$ is odd.}
  \end{cases}
\]
is !!{surjective}, but not !!{injective}.
\end{ex}

\begin{explain}
Often enough, we want to consider functions which are both
!!{injective} and !!{surjective}. We call such functions
!!{bijective}. They look like the function pictured in
\olref{fig:bijective}. !!^{bijection}s are also sometimes called
\emph{one-to-one correspondences}, since they uniquely pair elements
of the codomain with elements of the domain.
\begin{figure}
  \olasset{assets/diagrams/bijective.tikz}
  \caption{!!^a{bijective} function uniquely pairs the elements of the
    codomain with those of the domain.}
  \ollabel{fig:bijective}
\end{figure}
\end{explain}

\begin{defn}[!!^{bijection}] 
A function $f \colon A \to B$ is \emph{!!{bijective}} iff it is both
!!{surjective} and !!{injective}. We call such a function
!!a{bijection} from $A$ to~$B$ (or between $A$ and~$B$).
\end{defn}

\end{document}
