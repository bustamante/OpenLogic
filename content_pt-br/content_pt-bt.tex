% open-logic.tex
% compiles to produce complete text using open-logic-dev style

\documentclass[../include/open-logic-part]{subfiles}

\begin{document}

\clearpage

\begin{editorial}
This file loads all content included in the Open Logic Project.
Editorial notes like this, if displayed, indicate that the file was
compiled without any thought to how this material will be presented.
If you can read this, it is probably \emph{not advisable} to teach or
study from this PDF. 

The Open Logic Project provides many mechanisms by which a text can be
generate which is more appropriate for teaching or self-study.  For
instance, by default, the text will make all logical operators
primitives and carry out all cases for all operators in proofs.  But
it is much better to leave some of these cases as exercises. The Open
Logic Project is also a work in progress. In an effort to stimulate
collaboration and improvement, material is included even if it is
only in draft form, is missing exercises, etc.  A PDF produced for a
course will exclude these sections.

To find PDFs more suitable for teaching and studying, have a look at
the \href{http://builds.openlogicproject.org/}{sample courses
  available on the OLP website}. To make your own, you might start
from the
\href{https://github.com/OpenLogicProject/OpenLogic/tree/master/courses/sample}{sample driver file} or look at the sources of the derived textbooks for
more fancy and advanced examples.
\end{editorial}

\olimport[sets-functions-relations]{sets-functions-relations-complete}

%\olimport[propositional-logic]{propositional-logic}

%\olimport[first-order-logic]{first-order-logic}

%\olimport[model-theory]{model-theory}

%\olimport[computability]{computability}

%\olimport[turing-machines]{turing-machines}

%\olimport[incompleteness]{incompleteness}

%\olimport[second-order-logic]{second-order-logic}

%\olimport[lambda-calculus]{lambda-calculus}

%\olimport[normal-modal-logic]{normal-modal-logic}

%\olimport[intuitionistic-logic]{intuitionistic-logic}

%\olimport[counterfactuals]{counterfactuals}

%\olimport[set-theory]{set-theory}

%\olimport[methods]{methods}

%\olimport[history]{history}

\end{document}

